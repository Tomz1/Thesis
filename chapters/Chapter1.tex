\chapter{Introduction}
\label{chapter:introduction}
\lhead{Chapter 1. \emph{Introduction}}

%\section{Motivation and Framing}

Approximately 90\% of global trade relies on the international shipping industry. Consequently, the ocean is a vital platform for the world economy.
Currently, there are approximately 50.000 merchant ships trading internationally. Given the current demand, this number is bound to increase~\cite{ICS}. Not all such activity is legitimate, with some of it resorting to organised crime and various other illicit schemes that prevail in the maritime domain. Examples of this may be given by piracy, drug trafficking, illegal immigration, arms proliferation and illegal fishing. The definition of maritime safety is a complex endeavour and widely acknowledged as a transnational task~\cite{Bueger2015}.

Tracking people and objects within a geographical space has become a ubiquitous challenge. Automatic Identification System (AIS)\label{label_AIS} is an automated tracking system that broadcasts information through very high frequency (VHF) bands, which ultimately assist vessels in navigation. Imposed by the IMO (International Maritime Organization),\label{label_IMO} every SOLAS
(Safety of Life at Sea) \label{label_SOLAS} vessel must be equipped with such a device.
Autonomously broadcast AIS messages contain \emph{kinematic} information such as the ship location, speed, heading, rate of turn, destination and estimated arrival time, as well as \emph{static} information, including the ship name, ID, type, size. AIS messages can be transformed into useful information for maritime traffic manipulations such as vessel path prediction and collision avoidance. For these reasons, the AIS tracking system plays a central role within the development of future autonomous maritime navigation systems~\cite{Mao2016}.

The introduction of AIS in the maritime domain lead to an exponential increase of the volume of vessel trajectory data, making human analysis and evaluation of such data extremely inefficient. Therefore, new effective ways to automatically mine this data are of extreme importance for the future of nautical surveillance. Despite its advancements, mining maritime trajectory data still presents several challenges. Firstly, such data contains uncertainty typical of moving objects. Geo-referenced locations of trajectories constructed by location sensing techniques are prone to spatial uncertainty due to computational error and signal degradation or loss associated with the positioning device. Temporal uncertainty may be generated by different sampling rates and temporal lengths~\cite{Lee}. 
Secondly, maritime traffic is not constrained to roads - vessels are free to navigate in open waters as long as legal restrictions are observed. These situations hint at the inherent complexity of detecting trajectory anomalies.
Nevertheless, vessels tend to be observed travelling in the most economic route, to the advantage of shipping companies. This situation creates a behavioural baseline, from which anomalous behaviour may be inferred. This task reflects the main subject-matter shown in this work.

The definition of anomalous vessel behaviour is of paramount importance and it is given in Section~\ref{section: Framework Requirements}. Regardless of how such anomalies are construed, a framework capable of dealing with both the detection and identification of anomalous behaviour may be designed. A brief review of the various Anomalous Detection (AD) Frameworks found in the literature, alongside their scope variants, is presented in Chapter~\ref{chapter:literatureReview}. Such methods are tailored to different requirements, which are not always synchronised with the ones aimed for the particular needs of this work.

The work that is developed throughout this thesis is integrated within an ongoing highly-collaborative European project, the \textsc{Marisa} project~\footnote{https://marisaproject.eu}. Maritime Integrated Surveillance Awareness \textsc{Marisa} is a project funded by European commission under a Horizon 2020 research and innovation programme. The mission of the \textsc{Marisa} project is to enhance the decision making and reaction capabilities of the maritime authorities, by the development of a toolkit. 
This is achieved within 22 entities working collaboratively towards the current real-world demands of the maritime authorities, which ultimately are the end-user of the toolkit. Such demands were initially presented for the project by the maritime experts and grouped into two major set of activities. The first group of activities, representing also the first stage of the project, focus on the usage of state-of-the-art technologies towards the collaborative development of the \textsc{Marisa} toolkit. The second set of activities is related to the validation of the toolkit capabilities by the execution of trials across different end-user sites.  
Through the process of meeting with the \textsc{Marisa} end-users, the focus of our current work was defined. The objectives of this present were then focused on the first set of the project activities, with a higher emphasis on usage of novel techniques and algorithms to collect and properly process large amounts of heterogeneous data sets for early warning, forensic purposes and illegal act prosecution. Thus, for the sole-purpose of this work and given the context of the \textsc{Marisa} project, a set of specific bjectives were defined, and are presented under in Section~\ref{section: thesis objectives}.

\section{Objectives}
\label{section: thesis objectives}
%Given the particular needs for the MARISA project,
%We have developed a custom-built framework which takes into account the particular project requirements of \textsc{Marisa}. 
Taking into account the Inov tasks, for the specific objectives of this thesis, we are concerned with the task:

{\centering
\emph{
of developing a framework to take vast amounts of unstructured vessel data and, upon appropriate meaningful data structuring to be capable of recreating a vessel trajectory storing in a database as well as analysing information that ultimately allows for the detection of what is defined to be an anomaly}
}

The formalised objective is admittedly general and entails many technically distinct challenges, both conceptual and practical. To tackle such difficulties we subdivided the general objective into smaller objectives, thus breaking down a problem into smaller sub-problems: 

\begin{itemize}
\item Ingest, pre-process and structure high-throughput's of maritime data.
\item Provide procedures to transform spatial vessel data, into sequential data, thus defining vessel trajectory.
\item Develop anomaly detection methods, based on the what is to be defined vessel anomalous behaviour, by the competent entities.
\item Containerise the solution for the previous objectives into a framework which can be used into different maritime scenarios.  
\end{itemize}

\section{Outline}
Following the introduction, the remainder of this thesis is organised in 6 Chapters. A literature review, were questions regarding the maritime domain safety, and used technologies were explored. In the same Chapter we study the previous behaviour analysis frameworks presented in the literature. Methods for trajectory representation, regarding vessel trajectories are also discussed. Following this Chapter, in Chapter~\ref{chapter:Chapter 3}, we define what is to be considered an \emph{Behavioural Anomaly} for this thesis. Following, we introduce the developed Modular Anomaly Detection Framework (MAD-F), describing the purpose of each module, and the considered data types. Chapter~\ref{chapter:Chapter 4} we firstly provide an vessel data-set analysis, which was our initial contact with such domain specific data-types. Further, we explain the development and decisions took trough each modules of the \emph{MAD-F}. Chapter~\ref{chapter:Chapter 5} presents the results which were obtained per experiment. Finally Chapter~\ref{chapter:Chapter6}, discusses the limitations of the presented work and presents recommendations for future research.

\iffalse
The current research will maintain a similar structure throughout, dividing each chapter into two
parts; each addressing their assigned research question. Following the Introduction, Chapter 2 will
present the Literature Review, which will discuss and relate previous studies to the current research.
Chapter 3 presents the Method, which addresses the necessary methodology required per part. Part 1
can be considered an exploratory analysis of Bejarano’s (2016) work and of the data itself, whereas
part 2 briefly explains the new target feature and subset required to detect fishing gear types.
Naturally, the experiments will be explained in this chapter prior to conducting them. Chapter 4
presents the results obtained per experiment. Chapter 5 presents the Discussion, where the
classification results will be analyzed in further detail along with the contributions, limitations, and recommendations for future research.

The remainder of this work is organized as follows: Section 2 reviews the background
and related work regarding the open data, AD and data fusion in the MS domain. Section 3
presents the research methodology. Validity threats and verification are described in sections
4 and 5, respectively. Section 6 presents the experiment results and the validation results are
shown in section 7. Section 8 features a detailed discussion about the obtained results.
Finally, section 9 concludes the research with a discussion on the possible directions for
future work.
\fi

\iffalse
Before engaging in such technicalities, two general behavioural profiles can be identified:
\todo[inline]{kinematic nao dizer que e mais da traj...}
\begin{description}
\item[\emph{Kinematic} behaviours] relate to the motion of ships including the routes taken and speed of travel.
\item [AIS transmission behaviours] include the switching on or off of AIS systems and changing the vessels name or other details. Other behaviours such as changes in crew members, or ship registration details could also be categorised~\cite{Lane2010}.
%On this work the focus will be on anomalous behavior that can be described using AIS data.
\end{description}


These two classes present inherently different challenges in their identification and detection, given their nature. From a functional point of view, they are both vital in inferring that anomalous behaviour has been observed. This 

In chapter ~\ref{chapter:literatureReview} a solid state of the art is presented, with the emphases on methods and algorithms that mine trajectories data, in order to achieve a system that satisfies the main research question.

\subsection{Problem identification and motivation:} 
The process of identifying vessels with an abnormal behaviour, is a complex task. This is mainly because in vessels, abnormal behaviour can be caused by numerous different causes.
  
The authors ~\cite{Laxhammar2008} define Anomaly Detection (AD) as "a method for separating an often in-homogeneous and hard characterized minority of data from a more regular majority of data, by studying and characterizing the majority, so that data in the minority appears as deviations from the patterns found in the majority". 

The main focus of this work is the research and develop of methods that represent vessel motion data, in ways that anomalies can be found.
\fi
