\chapter{Conclusion and Future Work}
\label{chapter:Chapter6}
\lhead{Chapter 6. \emph{Conclusion and Future Work}}

In summary, this work is concerned with the detection and identification of anomalies at seas. This task in itself, as was discussed throughout this work, is rather complex and can only be meaningful when a suitable definition of anomaly is applied to the problem at hand. Without such particular constraints, both the aim and eventual conclusions become of general scope. Greatly justified by this reason, we were fortunate to have discussed these technicalities with Maritime experts via the \textsc{Marisa} project. Their insights and feedback lead to a workable interaction level which ultimately allowed us for the developmnet of a number of data-driven methods to be applied, resulting in useful and exciting conclusions....


Given the We defined such concep 
present The concept of anomaly detected, 


The proposed MAD-F is presented as a solution for this. Despite the acknowledged limitations from Modular Anomaly Detection Framework, the framework is 

The availability of large trajectory datasets in urban environments offers many new opportunities
for analyzing human displacements in space and time. A